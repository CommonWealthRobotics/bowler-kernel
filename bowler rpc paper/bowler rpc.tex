\documentclass{article}
\usepackage[utf8]{inputenc}
\usepackage{bytefield}
\usepackage{graphicx}
\usepackage{subcaption}
\usepackage{placeins}

\title{Bowler RPC}
\author{Ryan Benasutti}
\date{March 2019}

\begin{document}

\maketitle

\section{Discovery}

\subsection{Packet Format}

\subsubsection{General Discovery Packet Format}

Figure~\ref{fig:discovery-time-send-packet-format} shows what the PC sends the device to initiate a
discovery operation. Any additional operation-specific data is sent in the Payload section. The
entire packet is 64 bytes.

\begin{figure}[h]
    \centering
    \begin{bytefield}[]{16}
        \bitheader{0,7,8,15} \\
        \wordbox{2}{DPID} \\
        \bitbox{8}{Operation}
        \bitbox[lrt]{8}{} \\
        \wordbox[lr]{1}{Payload} \\
        \skippedwords \\
        \wordbox[lrb]{1}{}
    \end{bytefield}
    \caption{Discovery-time send packet format.}
    \label{fig:discovery-time-send-packet-format}
\end{figure}

\begin{itemize}
    \item DPID (Discovery Packet ID): 4 bytes
    \begin{itemize}
        \item The DPID field is typically filled by SimplePacketComs and contains the ID for the
        packet it is contained in.
    \end{itemize}

    \item Operation: 1 byte
    \begin{itemize}
        \item The Operation field states the operation the packet performs.
    \end{itemize}
\end{itemize}

Figure~\ref{fig:discovery-time-receive-packet-format} shows what the device sends the PC to complete
a discovery operation. Any additional operation-specific data is sent in the Payload section. The
entire packet is 64 bytes.

\begin{figure}[h]
    \centering
    \begin{bytefield}[]{16}
        \bitheader{0,7,8,15} \\
        \wordbox{2}{DPID} \\
        \wordbox[lr]{1}{Payload} \\
        \skippedwords \\
        \wordbox[lrb]{1}{}
    \end{bytefield}
    \caption{Discovery-time receive packet format.}
    \label{fig:discovery-time-receive-packet-format}
\end{figure}

\FloatBarrier

\begin{itemize}
    \item DPID (Discovery Packet ID): 4 bytes
    \begin{itemize}
        \item The DPID field is typically filled by SimplePacketComs and contains the ID for the
        packet it is contained in.
    \end{itemize}
\end{itemize}

\subsubsection{Discovery Packet}

\begin{figure}[h]
    \centering
    \begin{bytefield}[]{16}
        \bitheader{0,7,8,15} \\
        \wordbox{2}{DPID} \\
        \bitbox{8}{1}
        \bitbox{8}{Packet ID} \\
        \begin{leftwordgroup}{ResourceId}
            \bitbox{8}{Resource}
            \bitbox{8}{Attachment} \\
            \wordbox[lr]{1}{Attachment Data} \\
            \skippedwords \\
            \wordbox[lrb]{1}{}
        \end{leftwordgroup}
    \end{bytefield}
    \caption{Discovery send packet.}
\end{figure}

\begin{itemize}
    \item DPID (Discovery Packet ID): 4 bytes
    \begin{itemize}
        \item The DPID field is typically filled by SimplePacketComs and contains the ID for the
        packet it is contained in.
    \end{itemize}

    \item Packet ID: 1 byte
    \begin{itemize}
        \item The Packet ID is a new ID for the Packet being discovered.
    \end{itemize}

    \item Resource: 1 byte
    \begin{itemize}
        \item The Resource encodes the type of the resource. It is the \\
        \texttt{ResourceId.resourceType.type}.
    \end{itemize}

    \item Attachment: 1 byte
    \begin{itemize}
        \item The Attachment encodes the type of the attachment point. It is the
        \texttt{ResourceId.attachmentPoint.type}.
    \end{itemize}

    \item Attachment Data: 1+ bytes
    \begin{itemize}
        \item The Attachment Data is any data needed to fully describe the Attachment. It is the
        \texttt{ResourceId.attachmentPoint.data}.
    \end{itemize}
\end{itemize}

\begin{figure}[h]
    \centering
    \begin{bytefield}[]{16}
        \bitheader{0,7,8,15} \\
        \wordbox{2}{DPID} \\
        \bitbox{8}{Status}
        \bitbox[lrt]{8}{} \\
        \wordbox[lr]{1}{Reserved} \\
        \skippedwords \\
        \wordbox[lrb]{1}{}
    \end{bytefield}
    \caption{Discovery receive packet.}
\end{figure}

\begin{itemize}
    \item DPID (Discovery Packet ID): 4 bytes
    \begin{itemize}
        \item The DPID field is typically filled by SimplePacketComs and contains the ID for the
        packet it is contained in.
    \end{itemize}

    \item Status: 1 byte
    \begin{itemize}
        \item The Status field encodes the status of the discovery operation. 1 = Accepted, 2 =
        Rejected.
    \end{itemize}
\end{itemize}

\end{document}
