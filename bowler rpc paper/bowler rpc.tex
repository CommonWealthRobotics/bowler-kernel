\documentclass{article}
\usepackage[utf8]{inputenc}
\usepackage{bytefield}
\usepackage{graphicx}
\usepackage{subcaption}
\usepackage{placeins}

\title{
    The Bowler RPC \\
    \large A zeroconf protocol for creating PC-microcontroller networks
}
\author{Ryan Benasutti \\ Common Wealth Robotics Cooperative}
\date{March 2019}

\begin{document}

\maketitle

\section{Overview}

\subsection{Motivation}

\begin{enumerate}
    \item We want a PC to talk to any Device which is running some Bowler software. We want
    this software to require minimal to no configuration from the user.
    
    \item We want to support any Resource attached to the Device with minimal to no configuration
    from the user.

    \item We require the communications between the PC and Device satisfy hard real-time
    requirements: 5ms RTT, 100ms timeout.

    \item Therefore, we settle on a zeroconf protocol which can be used by the PC to establish
    an RPC with the Device.
\end{enumerate}

\subsection{Configuration Process}

The PC-Device RPC is established in the following order
\begin{enumerate}
    \item The PC connects to the Device using some Physical Layer implementation.
    \item The PC sends Discovery packets to the Device to tell the device which Resources are
    connected to it. The Device may reject any packet if it deems the Resource invalid.
    \item Once all Resources have been discovered, the Discovery process is finished and the
    PC and Device may use the configured RPC.
\end{enumerate}

\section{Discovery}

\subsection{Packet Format}

\subsubsection{General Discovery Packet Format}

Figure~\ref{fig:discovery-time-send-packet-format} shows what the PC sends the device to initiate a
discovery operation. Any additional operation-specific data is sent in the Payload section. The
entire packet is 64 bytes.

\begin{figure}[h]
    \centering
    \begin{bytefield}[]{16}
        \bitheader{0,7,8,15} \\
        \wordbox{2}{DPID} \\
        \bitbox{8}{Operation}
        \bitbox[lrt]{8}{} \\
        \wordbox[lr]{1}{Payload} \\
        \skippedwords \\
        \wordbox[lrb]{1}{}
    \end{bytefield}
    \caption{Discovery-time send packet format.}
    \label{fig:discovery-time-send-packet-format}
\end{figure}

\FloatBarrier

\begin{itemize}
    \item DPID (Discovery Packet ID): 4 bytes
    \begin{itemize}
        \item The DPID field is typically filled by SimplePacketComs and contains the ID for the
        packet it is contained in.
    \end{itemize}

    \item Operation: 1 byte
    \begin{itemize}
        \item The Operation field states the operation the packet performs.
    \end{itemize}
\end{itemize}

\FloatBarrier

Figure~\ref{fig:discovery-time-receive-packet-format} shows what the device sends the PC to complete
a discovery operation. Any additional operation-specific data is sent in the Payload section. The
entire packet is 64 bytes.

\begin{figure}[h]
    \centering
    \begin{bytefield}[]{16}
        \bitheader{0,7,8,15} \\
        \wordbox{2}{DPID} \\
        \wordbox[lr]{1}{Payload} \\
        \skippedwords \\
        \wordbox[lrb]{1}{}
    \end{bytefield}
    \caption{Discovery-time receive packet format.}
    \label{fig:discovery-time-receive-packet-format}
\end{figure}

\FloatBarrier

\begin{itemize}
    \item DPID (Discovery Packet ID): 4 bytes
    \begin{itemize}
        \item The DPID field is typically filled by SimplePacketComs and contains the ID for the
        packet it is contained in.
    \end{itemize}
\end{itemize}

\FloatBarrier

\subsubsection{Discovery Packet}

\begin{figure}[h]
    \centering
    \begin{bytefield}[]{16}
        \bitheader{0,7,8,15} \\
        \wordbox{2}{DPID} \\
        \bitbox{8}{1}
        \bitbox{8}{Packet ID} \\
        \begin{leftwordgroup}{ResourceId}
            \bitbox{8}{Resource}
            \bitbox{8}{Attachment} \\
            \wordbox[lr]{1}{Attachment Data} \\
            \skippedwords \\
            \wordbox[lrb]{1}{}
        \end{leftwordgroup}
    \end{bytefield}
    \caption{Discovery send packet.}
\end{figure}

\FloatBarrier

\begin{itemize}
    \item DPID (Discovery Packet ID): 4 bytes
    \begin{itemize}
        \item The DPID field is typically filled by SimplePacketComs and contains the ID for the
        packet it is contained in.
    \end{itemize}

    \item Packet ID: 1 byte
    \begin{itemize}
        \item The Packet ID field is a new ID for the Packet being discovered.
    \end{itemize}

    \item Resource: 1 byte
    \begin{itemize}
        \item The Resource field is the type of the resource. It is the \\
        \texttt{ResourceId.resourceType.type}.
    \end{itemize}

    \item Attachment: 1 byte
    \begin{itemize}
        \item The Attachment field is the type of the attachment point. It is the
        \texttt{ResourceId.attachmentPoint.type}.
    \end{itemize}

    \item Attachment Data: 1+ bytes
    \begin{itemize}
        \item The Attachment Data field is any data needed to fully describe the Attachment. It is
        the \texttt{ResourceId.attachmentPoint.data}.
    \end{itemize}
\end{itemize}

\FloatBarrier

\begin{figure}[h]
    \centering
    \begin{bytefield}[]{16}
        \bitheader{0,7,8,15} \\
        \wordbox{2}{DPID} \\
        \bitbox{8}{Status}
        \bitbox[lrt]{8}{} \\
        \wordbox[lr]{1}{Reserved} \\
        \skippedwords \\
        \wordbox[lrb]{1}{}
    \end{bytefield}
    \caption{Discovery receive packet.}
\end{figure}

\FloatBarrier

\begin{itemize}
    \item DPID (Discovery Packet ID): 4 bytes
    \begin{itemize}
        \item The DPID field is typically filled by SimplePacketComs and contains the ID for the
        packet it is contained in.
    \end{itemize}

    \item Status: 1 byte
    \begin{itemize}
        \item The Status field encodes the status of the discovery operation. 1 = Accepted, 2 =
        Rejected.
    \end{itemize}
\end{itemize}

\FloatBarrier

\subsubsection{Group Discovery Packet}

\begin{figure}[h]
    \centering
    \begin{bytefield}[]{16}
        \bitheader{0,7,8,15} \\
        \wordbox{2}{DPID} \\
        \bitbox{8}{3}
        \bitbox{8}{Group ID} \\
        \bitbox{8}{Packet ID}
        \bitbox{8}{Count} \\
        \wordbox[lr]{1}{Reserved} \\
        \skippedwords \\
        \wordbox[lrb]{1}{}
    \end{bytefield}
    \caption{Group discovery send packet.}
\end{figure}

\FloatBarrier

\begin{itemize}
    \item DPID (Discovery Packet ID): 4 bytes
    \begin{itemize}
        \item The DPID field is typically filled by SimplePacketComs and contains the ID for the
        packet it is contained in.
    \end{itemize}

    
    \item Group ID: 1 byte
    \begin{itemize}
        \item The Group ID field is the ID for the group being made. Future Group Member Discovery
        Packets will need this ID to add Resources to the correct group.
    \end{itemize}

    \item Packet ID: 1 byte
    \begin{itemize}
        \item The Packet ID field is the ID for the packet the Group will use. All Resources in the
        Group get packed into one packet.
    \end{itemize}

    \item Count: 1 byte
    \begin{itemize}
        \item The Count field is the number of Resources that will be added to the group.
    \end{itemize}
\end{itemize}

\FloatBarrier

\begin{figure}[h]
    \centering
    \begin{bytefield}[]{16}
        \bitheader{0,7,8,15} \\
        \wordbox{2}{DPID} \\
        \bitbox{8}{Status}
        \bitbox[lrt]{8}{} \\
        \wordbox[lr]{1}{Reserved} \\
        \skippedwords \\
        \wordbox[lrb]{1}{}
    \end{bytefield}
    \caption{Group discovery receive packet.}
\end{figure}

\FloatBarrier

\begin{itemize}
    \item DPID (Discovery Packet ID): 4 bytes
    \begin{itemize}
        \item The DPID field is typically filled by SimplePacketComs and contains the ID for the
        packet it is contained in.
    \end{itemize}

    \item Status: 1 byte
    \begin{itemize}
        \item The Status field encodes the status of the discovery operation. 1 = Accepted, 2 =
        Rejected.
    \end{itemize}
\end{itemize}

\FloatBarrier

\subsubsection{Group Member Discovery Packet}

\begin{figure}[h]
    \centering
    \begin{bytefield}[]{16}
        \bitheader{0,7,8,15} \\
        \wordbox{2}{DPID} \\
        \bitbox{8}{4}
        \bitbox{8}{Group ID} \\
        \bitbox{8}{Start}
        \bitbox{8}{End} \\
        \begin{leftwordgroup}{ResourceId}
            \bitbox{8}{Resource}
            \bitbox{8}{Attachment} \\
            \wordbox[lr]{1}{Attachment Data} \\
            \skippedwords \\
            \wordbox[lrb]{1}{}
        \end{leftwordgroup}
    \end{bytefield}
    \caption{Group member discovery send packet.}
\end{figure}

\FloatBarrier

\begin{itemize}
    \item Group ID: 1 byte
    \begin{itemize}
        \item The Group ID field is the ID for the Group that this Resource will be added to.
    \end{itemize}

    \item Start: 1 byte
    \begin{itemize}
        \item The Start field is the starting byte index in the receive Payload for this Resource's
        receive data.
    \end{itemize}

    \item End: 1 byte
    \begin{itemize}
        \item The End field is the ending byte index in the receive Payload for this Resource's
        receive data.
    \end{itemize}

    \item Resource: 1 byte
    \begin{itemize}
        \item The Resource field is the type of the resource. It is the \\
        \texttt{ResourceId.resourceType.type}.
    \end{itemize}

    \item Attachment: 1 byte
    \begin{itemize}
        \item The Attachment field is the type of the attachment point. It is the
        \texttt{ResourceId.attachmentPoint.type}.
    \end{itemize}

    \item Attachment Data: 1+ bytes
    \begin{itemize}
        \item The Attachment Data field is any data needed to fully describe the Attachment. It is
        the \texttt{ResourceId.attachmentPoint.data}.
    \end{itemize}
\end{itemize}

\FloatBarrier

\begin{figure}[h]
    \centering
    \begin{bytefield}[]{16}
        \bitheader{0,7,8,15} \\
        \wordbox{2}{DPID} \\
        \bitbox{8}{Status}
        \bitbox[lrt]{8}{} \\
        \wordbox[lr]{1}{Reserved} \\
        \skippedwords \\
        \wordbox[lrb]{1}{}
    \end{bytefield}
    \caption{Group member discovery receive packet.}
\end{figure}

\FloatBarrier

\begin{itemize}
    \item DPID (Discovery Packet ID): 4 bytes
    \begin{itemize}
        \item The DPID field is typically filled by SimplePacketComs and contains the ID for the
        packet it is contained in.
    \end{itemize}

    \item Status: 1 byte
    \begin{itemize}
        \item The Status field encodes the status of the discovery operation. 1 = Accepted, 2 =
        Rejected.
    \end{itemize}
\end{itemize}

\FloatBarrier

\subsection{Discovery Process}

\subsubsection{Discovery}

Sequence diagram for
\begin{enumerate}
    \item Send discovery packet and get response (accepted).
    \item Send discovery packet and get response (rejected).
\end{enumerate}

\subsubsection{Group Discovery}

Sequence diagram for
\begin{enumerate}
    \item Send group discovery packet and get response (accepted). Send multiple group member
    discovery packets and get responses (accepted).
    \item Send group discovery packet and get response (accepted). Send multiple group member
    discovery packets and get responses (most accepted, some rejected).
    \item Send group discovery packet and get response (rejected). Send multiple group member
    discovery packets and get responses (rejected).
\end{enumerate}

\section{RPC}

\subsection{Packet Format}

\subsubsection{Non-Group}

Packets for non-Group Resources correspond to a single Resource. These Resources do not have any
timing constraints.

\begin{figure}[h]
    \centering
    \begin{bytefield}[]{16}
        \bitheader{0,7,8,15} \\
        \wordbox{2}{DPID} \\
        \wordbox[lr]{1}{Payload} \\
        \skippedwords \\
        \wordbox[lrb]{1}{}
    \end{bytefield}
    \caption{RPC send packet format.}
    \label{fig:rpc-send-packet-format}
\end{figure}

\FloatBarrier

\begin{itemize}
    \item DPID (Discovery Packet ID): 4 bytes
    \begin{itemize}
        \item The DPID field is typically filled by SimplePacketComs and contains the ID for the
        packet it is contained in.
    \end{itemize}
\end{itemize}

\FloatBarrier

\begin{figure}[h]
    \centering
    \begin{bytefield}[]{16}
        \bitheader{0,7,8,15} \\
        \wordbox{2}{DPID} \\
        \wordbox[lr]{1}{Payload} \\
        \skippedwords \\
        \wordbox[lrb]{1}{}
    \end{bytefield}
    \caption{RPC receive packet format.}
    \label{fig:rpc-receive-packet-format}
\end{figure}

\FloatBarrier

\begin{itemize}
    \item DPID (Discovery Packet ID): 4 bytes
    \begin{itemize}
        \item The DPID field is typically filled by SimplePacketComs and contains the ID for the
        packet it is contained in.
    \end{itemize}
\end{itemize}

\FloatBarrier

\subsubsection{Group}

Packets for Group Resources correspond to multiple Resources whose responses are packed into a
single packet. These Resources can have timing constraints and are therefore put into a Group.

\begin{figure}[h]
    \centering
    \begin{bytefield}[]{16}
        \bitheader{0,7,8,15} \\
        \wordbox{2}{DPID} \\
        \wordbox[lr]{1}{Payload} \\
        \skippedwords \\
        \wordbox[lrb]{1}{}
    \end{bytefield}
    \caption{RPC send packet format.}
    \label{fig:rpc-send-packet-format}
\end{figure}

\FloatBarrier

\begin{itemize}
    \item DPID (Discovery Packet ID): 4 bytes
    \begin{itemize}
        \item The DPID field is typically filled by SimplePacketComs and contains the ID for the
        packet it is contained in.
    \end{itemize}
\end{itemize}

\FloatBarrier

\begin{figure}[h]
    \centering
    \begin{bytefield}[]{16}
        \bitheader{0,7,8,15} \\
        \wordbox{2}{DPID} \\
        \wordbox[lr]{1}{Payload} \\
        \skippedwords \\
        \wordbox[lrb]{1}{} \\
        \wordbox[lr]{1}{Payload} \\
        \skippedwords \\
        \wordbox[lrb]{1}{}
    \end{bytefield}
    \caption{RPC receive packet format.}
    \label{fig:rpc-receive-packet-format}
\end{figure}

\FloatBarrier

\begin{itemize}
    \item DPID (Discovery Packet ID): 4 bytes
    \begin{itemize}
        \item The DPID field is typically filled by SimplePacketComs and contains the ID for the
        packet it is contained in.
    \end{itemize}
\end{itemize}

\FloatBarrier

\end{document}
