\documentclass{article}
\usepackage[utf8]{inputenc}
\usepackage{bytefield}
\usepackage{graphicx}
\usepackage{subcaption}
\usepackage{placeins}
\usepackage[acronym, toc]{glossaries}
\usepackage{pgf-umlsd}

\makenoidxglossaries{}

\newglossaryentry{pc}
{
    name=PC,
    description={The host computer running the Bowler stack}
}

\newglossaryentry{device}
{
    name=device,
    description={A microcontroller running an implementation of the Bowler RPC}
}

\newglossaryentry{resource}
{
    name=resource,
    description={A hardware element connected to a \Gls{device}; typically an actuator or sensor}
}

\newglossaryentry{group}
{
    name=group,
    description={A collection of \Glspl{resource} which must be written to/read from at the same
    exact time.}
}

\newglossaryentry{discovery}
{
    name=discovery,
    description={A procedure in which the \Gls{pc} tells the \Gls{device} which \Glspl{resource} are
    connected to it}
}

\newglossaryentry{group-discovery}
{
    name=group discovery,
    description={A procedure in which the \Gls{pc} tells the \Gls{device} about a \Gls{group}}
}

\newglossaryentry{group-member-discovery}
{
    name=group member discovery,
    description={A procedure in which the \Gls{pc} tells the \Gls{device} about a
    \Gls{resource} which is part of a \Gls{group}}
}

\newglossaryentry{zeroconf}
{
    name=zeroconf,
    description={Zero-configuration; the automatic creation a computer network}
}

\newacronym{rpc}{RPC}{Remote Prodecure Call}

\newacronym{rtt}{RTT}{Round Trip Time}


\title{
    The Bowler RPC \\
    \large A zeroconf protocol for creating PC-microcontroller networks
}
\author{Ryan Benasutti \\ Common Wealth Robotics Cooperative}
\date{March 2019}

\begin{document}

\maketitle

\section{Overview}
\subsection{Motivation}

\begin{itemize}
    \item We want a \gls{pc} to talk to any \gls{device} which is running some Bowler software. We
    want this software to require minimal to no configuration from the user.
    
    \item We want to support any \gls{resource} attached to the \gls{device} with minimal to no
    configuration from the user.
\end{itemize}

\clearpage
\tableofcontents

\clearpage
\section{Requirements}

\subsection{Project Goals}

The goals of this project are as follows, in order of importance:

\begin{enumerate}
    \item Support real-time \gls{pc}-\gls{device} communication with \gls{rtt} $\leq$ 5ms, timeout
    $\leq$ 100ms. This requirement is the minimum, implementations ideally have a lower \gls{rtt}
    depending on the physical layer used, e.g. an HID-based implementation should tend to be faster
    and have less jitter than a UDP-based implementation.

    \item Support packet \glspl{group}. For example, two quadrature encoders on opposite sides of a
    skid-steer drivetrain must be read from at the exact same time; it is not possible to split
    these two reads into two packets.

    \item Optional RPC packet ordering. Some RPC packets (typically writes) need strong ordering.
    Some physical layers (e.g. UDP) do not guarantee packet ordering.

    \item Support dynamic configuration of \glspl{device} and \glspl{resource} and support
    dynamically resetting any configured state on the \gls{device}.

    \item Support a wide variety of \glspl{device} and \glspl{resource} without requiring
    \gls{device}-side code modifications.
\end{enumerate}

\subsection{Network Requirements}

This protocol makes the following assumptions about the network implementation it runs on top of:
\begin{itemize}
    \item Data corruption is handled. Corrupt packets can be corrected or dropped.

    \item No packet can be duplicated in the network.
\end{itemize}

\subsection{Configuration Process}

The \gls{pc}-\gls{device} \gls{rpc} is established in the following order
\begin{enumerate}
    \item The \gls{pc} connects to the \gls{device} using some physical layer.
    
    \item The \gls{pc} sends \gls{discovery} packets to the \gls{device} to tell the \gls{device} which
    \glspl{resource} are connected to it. The \gls{device} may reject any packet if it deems the
    \gls{resource} invalid.
    
    \item Once all \glspl{resource} have been discovered, the \gls{discovery} process is finished and the
    \gls{pc} and \gls{device} may use the configured \gls{rpc}.

    \item At any time, the \gls{pc} may tell the \gls{device} to discard everything which was
    allocated or configured during \gls{discovery}. The \gls{device} must then be ready to start the
    \gls{discovery} process again.
\end{enumerate}

\clearpage
\section{Discovery}

Before the \gls{pc} can communicate with the \gls{device}, it must first configure the \gls{device}
using the \gls{discovery} process. This section describes that process and the packets involved.

\subsection{Packet Format}
\subsubsection{General Discovery Packet Format}

Figure~\ref{fig:discovery-time-send-packet-format} shows what the \gls{pc} sends the \gls{device} to
run \gls{discovery}. Any additional \gls{operation}-specific data is sent in the Payload section.
The entire packet is 60 bytes. Typically, this protocol is implemented using
\texttt{SimplePacketComs}, which uses the first 4 bytes of the packet for its header, leaving this
protocol with 60 bytes.

\begin{figure}[h]
    \centering
    \begin{bytefield}{16}
        \bitheader{0,7,8,15} \\
        \bitbox{8}{Operation}
        \bitbox[lrt]{8}{} \\
        \wordbox[lr]{1}{Payload} \\
        \skippedwords \\
        \wordbox[lrb]{1}{}
    \end{bytefield}
    \caption{Discovery-time send packet format.}
    \label{fig:discovery-time-send-packet-format}
\end{figure}

\FloatBarrier{}

\begin{itemize}
    \item Operation: 1 byte
    \begin{itemize}
        \item The Operation field states the \gls{operation} the packet performs.
    \end{itemize}
\end{itemize}

\FloatBarrier{}

Figure~\ref{fig:discovery-time-receive-packet-format} shows what the \gls{device} sends the \gls{pc}
to complete \gls{discovery}. Any additional operation-specific data is sent in the Payload section.
The entire packet is 60 bytes. Typically, this protocol is implemented using
\texttt{SimplePacketComs}, which uses the first 4 bytes of the packet for its header, leaving this
protocol with 60 bytes.

\begin{figure}[h]
    \centering
    \begin{bytefield}{16}
        \bitheader{0,7,8,15} \\
        \bitbox{8}{Status}
        \bitbox[lrt]{8}{} \\
        \wordbox[lr]{1}{Payload} \\
        \skippedwords \\
        \wordbox[lrb]{1}{}
    \end{bytefield}
    \caption{Discovery-time receive packet format.}
    \label{fig:discovery-time-receive-packet-format}
\end{figure}

\FloatBarrier{}

\begin{itemize}
    \item Status: 1 byte
    \begin{itemize}
        \item The Status field encodes the status of the \gls{discovery} operation. The status codes
        are documented in Table~\ref{tab:status-codes}.
    \end{itemize}
\end{itemize}

\FloatBarrier{}
\clearpage
\subsubsection{Discovery Packet}

\begin{figure}[h]
    \centering
    \begin{bytefield}{16}
        \bitheader{0,7,8,15} \\
        \bitbox{8}{1}
        \bitbox{8}{Packet ID} \\
        \begin{leftwordgroup}{\texttt{ResourceId}}
            \bitbox{8}{Resource}
            \bitbox{8}{Attachment} \\
            \wordbox[lr]{1}{Attachment Data} \\
            \skippedwords \\
            \wordbox[lrb]{1}{}
        \end{leftwordgroup}
    \end{bytefield}
    \caption{Discovery send packet.}
    \label{fig:discovery-send-packet}
\end{figure}

\FloatBarrier{}

\begin{itemize}
    \item Packet ID: 1 byte
    \begin{itemize}
        \item The Packet ID field is a new ID for the packet being discovered.
    \end{itemize}

    \item Resource: 1 byte
    \begin{itemize}
        \item The Resource field is the type of the \gls{resource}. It is the \\
        \texttt{ResourceId.resourceType.type}.
    \end{itemize}

    \item Attachment: 1 byte
    \begin{itemize}
        \item The Attachment field is the type of the attachment point. It is the
        \texttt{ResourceId.attachmentPoint.type}.
    \end{itemize}

    \item Attachment Data: 1+ bytes
    \begin{itemize}
        \item The Attachment Data field is any data needed to fully describe the Attachment. It is
        the \texttt{ResourceId.attachmentPoint.data}.
    \end{itemize}
\end{itemize}

\FloatBarrier{}

\begin{figure}[h]
    \centering
    \begin{bytefield}{16}
        \bitheader{0,7,8,15} \\
        \bitbox{8}{Status}
        \bitbox[lrt]{8}{} \\
        \wordbox[lr]{1}{Reserved} \\
        \skippedwords \\
        \wordbox[lrb]{1}{}
    \end{bytefield}
    \caption{Discovery receive packet.}
    \label{fig:discovery-receive-packet}
\end{figure}

\FloatBarrier{}

\begin{itemize}
    \item Status: 1 byte
    \begin{itemize}
        \item The Status field encodes the status of the \gls{discovery} operation. The status codes
        are documented in Table~\ref{tab:status-codes}.
    \end{itemize}
\end{itemize}

\FloatBarrier{}
\clearpage
\subsubsection{Group Discovery Packet}

\begin{figure}[h]
    \centering
    \begin{bytefield}{16}
        \bitheader{0,7,8,15} \\
        \bitbox{8}{2}
        \bitbox{8}{Group ID} \\
        \bitbox{8}{Packet ID}
        \bitbox{8}{Count} \\
        \wordbox[lr]{1}{Reserved} \\
        \skippedwords \\
        \wordbox[lrb]{1}{}
    \end{bytefield}
    \caption{Group discovery send packet.}
    \label{fig:group-discovery-send-packet}
\end{figure}

\FloatBarrier{}

\begin{itemize}
    \item Group ID: 1 byte
    \begin{itemize}
        \item The Group ID field is the ID for the \gls{group} being made. Future \gls{group-member-discovery}
        packets will need this ID to add \glspl{resource} to the correct \gls{group}.
    \end{itemize}

    \item Packet ID: 1 byte
    \begin{itemize}
        \item The Packet ID field is the ID for the packet the \gls{group} will use. All \glspl{resource}
        in the \gls{group} get packed into one packet.
    \end{itemize}

    \item Count: 1 byte
    \begin{itemize}
        \item The Count field is the number of \glspl{resource} that will be added to the
        \gls{group}.
    \end{itemize}
\end{itemize}

\FloatBarrier{}

\begin{figure}[h]
    \centering
    \begin{bytefield}{16}
        \bitheader{0,7,8,15} \\
        \bitbox{8}{Status}
        \bitbox[lrt]{8}{} \\
        \wordbox[lr]{1}{Reserved} \\
        \skippedwords \\
        \wordbox[lrb]{1}{}
    \end{bytefield}
    \caption{Group discovery receive packet.}
    \label{fig:group-discovery-receive-packet}
\end{figure}

\FloatBarrier{}

\begin{itemize}
    \item Status: 1 byte
    \begin{itemize}
        \item The Status field encodes the status of the \gls{discovery} operation. The status codes
        are documented in Table~\ref{tab:status-codes}.
    \end{itemize}
\end{itemize}

\FloatBarrier{}
\clearpage
\subsubsection{Group Member Discovery Packet}

\begin{figure}[h]
    \centering
    \begin{bytefield}{16}
        \bitheader{0,7,8,15} \\
        \bitbox{8}{3}
        \bitbox{8}{Group ID} \\
        \begin{leftwordgroup}{Write Indices}
            \bitbox{8}{Send Start}
            \bitbox{8}{Send End}
        \end{leftwordgroup} \\
        \begin{leftwordgroup}{Read Indices}
            \bitbox{8}{Receive Start}
            \bitbox{8}{Receive End}
        \end{leftwordgroup} \\
        \begin{leftwordgroup}{\texttt{ResourceId}}
            \bitbox{8}{Resource}
            \bitbox{8}{Attachment} \\
            \wordbox[lr]{1}{Attachment Data} \\
            \skippedwords \\
            \wordbox[lrb]{1}{}
        \end{leftwordgroup}
    \end{bytefield}
    \caption{Group member discovery send packet.}
    \label{fig:group-member-discovery-send-packet}
\end{figure}

\FloatBarrier{}

\begin{itemize}
    \item Group ID: 1 byte
    \begin{itemize}
        \item The Group ID field is the ID for the \gls{group} that this \gls{resource} will be added to.
    \end{itemize}

    \item Send Start: 1 byte
    \begin{itemize}
        \item The Send Start field is the starting byte index in the send Payload for this
        \gls{resource}'s write data (i.e. from the \gls{pc} perspective).
    \end{itemize}

    \item Send End: 1 byte
    \begin{itemize}
        \item The Send End field is the ending byte index in the send Payload for this \gls{resource}'s
        write data (i.e. from the \gls{pc} perspective).
    \end{itemize}

    \item Receive Start: 1 byte
    \begin{itemize}
        \item The Receive Start field is the starting byte index in the receive Payload for this
        \gls{resource}'s read data (i.e. from the \gls{pc} perspective).
    \end{itemize}

    \item Receive End: 1 byte
    \begin{itemize}
        \item The Receive End field is the ending byte index in the receive Payload for this
        \gls{resource}'s read data (i.e. from the \gls{pc} perspective).
    \end{itemize}

    \item Resource: 1 byte
    \begin{itemize}
        \item The Resource field is the type of the \gls{resource}. It is the \\
        \texttt{ResourceId.resourceType.type}.
    \end{itemize}

    \item Attachment: 1 byte
    \begin{itemize}
        \item The Attachment field is the type of the attachment point. It is the
        \texttt{ResourceId.attachmentPoint.type}.
    \end{itemize}

    \item Attachment Data: 1+ bytes
    \begin{itemize}
        \item The Attachment Data field is any data needed to fully describe the Attachment. It is
        the \texttt{ResourceId.attachmentPoint.data}.
    \end{itemize}
\end{itemize}

\FloatBarrier{}

\begin{figure}[h]
    \centering
    \begin{bytefield}{16}
        \bitheader{0,7,8,15} \\
        \bitbox{8}{Status}
        \bitbox[lrt]{8}{} \\
        \wordbox[lr]{1}{Reserved} \\
        \skippedwords \\
        \wordbox[lrb]{1}{}
    \end{bytefield}
    \caption{Group member discovery receive packet.}
    \label{fig:group-member-discovery-receive-packet}
\end{figure}

\FloatBarrier{}

\begin{itemize}
    \item Status: 1 byte
    \begin{itemize}
        \item The Status field encodes the status of the \gls{discovery} operation. The status codes
        are documented in Table~\ref{tab:status-codes}.
    \end{itemize}
\end{itemize}

\FloatBarrier{}
\clearpage
\subsubsection{Discard Discovery Packet}

At any time, the \gls{pc} may tell the \gls{device} to discard everything which was allocated or
configured during \gls{discovery}. The \gls{device} must then be ready to start the \gls{discovery}
process again. To initiate this process, the \gls{pc} may send the \gls{device} a discard
\gls{discovery} packet and continue polling the \gls{device} until it reports that the discard
process is complete.

\begin{figure}[h]
    \centering
    \begin{bytefield}{16}
        \bitheader{0,7,8,15} \\
        \bitbox{8}{4}
        \bitbox[lrt]{8}{} \\
        \wordbox[lr]{1}{Reserved} \\
        \skippedwords \\
        \wordbox[lrb]{1}{}
    \end{bytefield}
    \caption{Discard discovery send packet.}
    \label{fig:discard-discovery-send-packet}
\end{figure}

\FloatBarrier{}

\begin{figure}[h]
    \centering
    \begin{bytefield}{16}
        \bitheader{0,7,8,15} \\
        \bitbox{8}{Status}
        \bitbox[lrt]{8}{} \\
        \wordbox[lr]{1}{Reserved} \\
        \skippedwords \\
        \wordbox[lrb]{1}{}
    \end{bytefield}
    \caption{Discard discovery receive packet.}
    \label{fig:discard-discovery-receive-packet}
\end{figure}

\FloatBarrier{}

\begin{itemize}
    \item Status: 1 byte
    \begin{itemize}
        \item The Status field indicates the status of the discard operation. The status codes are
        documented in Table~\ref{tab:status-codes}.
    \end{itemize}
\end{itemize}

\FloatBarrier{}
\clearpage
\subsection{Discovery Process}
\subsubsection{Discovery}

\begin{figure}[h]
    \centering
    \begin{sequencediagram}
        \newthread{pc}{PC}
        \newinst[3]{device}{Device}

        \begin{call}{pc}{Send discovery}{device}{Status 1}
        \end{call}
    \end{sequencediagram}
    \caption{Successful discovery.}
    \label{fig:successful-discovery}
\end{figure}

\begin{figure}[h]
    \centering
    \begin{sequencediagram}
        \newthread{pc}{PC}
        \newinst[3]{device}{Device}

        \begin{call}{pc}{Send discovery}{device}{Status 2}
        \end{call}
    \end{sequencediagram}
    \caption{Unsuccessful discovery.}
    \label{fig:unsuccessful-discovery}
\end{figure}

\FloatBarrier{}
\clearpage
\subsubsection{Group Discovery}

\begin{figure}[h]
    \centering
    \begin{sequencediagram}
        \newthread{pc}{PC}
        \newinst[4]{device}{Device}

        \begin{call}{pc}{Send group discovery}{device}{Status 1}
        \end{call}

        \begin{call}{pc}{Send group member discovery}{device}{Status 1}
        \end{call}

        \begin{call}{pc}{Send group member discovery}{device}{Status 1}
        \end{call}
    \end{sequencediagram}
    \caption{Successful group discovery.}
    \label{fig:successful-group-discovery}
\end{figure}

\begin{figure}[h]
    \centering
    \begin{sequencediagram}
        \newthread{pc}{PC}
        \newinst[4]{device}{Device}

        \begin{call}{pc}{Send group discovery}{device}{Status 2}
        \end{call}

        \begin{call}{pc}{Send group member discovery}{device}{Status 7}
        \end{call}
    \end{sequencediagram}
    \caption{Unsuccessful group discovery.}
    \label{fig:unsuccessful-group-discovery}
\end{figure}

\FloatBarrier{}
\clearpage
\subsubsection{Discard Discovery}

\begin{figure}[h]
    \centering
    \begin{sequencediagram}
        \newthread{pc}{PC}
        \newinst[4]{device}{Device}

        \begin{call}{pc}{Send discard discovery}{device}{Status 10}
        \end{call}

        \begin{call}{pc}{Send discard discovery}{device}{Status 10}
        \end{call}

        \begin{call}{pc}{Send discard discovery}{device}{Status 11}
        \end{call}

        \begin{call}{pc}{Send discard discovery}{device}{Status 11}
        \end{call}
    \end{sequencediagram}
    \caption{Discard discovery.}
    \label{fig:discard-discovery}
\end{figure}

\FloatBarrier{}
\clearpage
\section{RPC}

After the \gls{discovery} process is complete and all \glspl{resource} have been discovered, the
\gls{pc} may begin "talking" with the \gls{device}. This section describes the format of the
configured packets.

\subsection{Packet Format}
\subsubsection{Non-Group}

Packets for non-\gls{group} \glspl{resource} correspond to a single \gls{resource}. These \glspl{resource}
do not have any timing constraints.

The non-\gls{group} send packet format consists of the write payload.

\begin{figure}[h]
    \centering
    \begin{bytefield}{16}
        \bitheader{0,7,8,15} \\
        \begin{leftwordgroup}{Resource Write}
            \wordbox[lrt]{1}{Payload} \\
            \skippedwords \\
            \wordbox[lrb]{1}{}
        \end{leftwordgroup}
    \end{bytefield}
    \caption{Non-group RPC send packet format.}
    \label{fig:non-group-rpc-send-packet-format}
\end{figure}

\FloatBarrier{}

The non-\gls{group} receive packet format consists of the read payload.

\begin{figure}[h]
    \centering
    \begin{bytefield}{16}
        \bitheader{0,7,8,15} \\
        \begin{leftwordgroup}{Resource Read}
            \wordbox[lrt]{1}{Payload} \\
            \skippedwords \\
            \wordbox[lrb]{1}{}
        \end{leftwordgroup}
    \end{bytefield}
    \caption{Non-group RPC receive packet format.}
    \label{fig:non-group-rpc-receive-packet-format}
\end{figure}

\FloatBarrier{}
\clearpage
\subsubsection{Group}

Packets for \gls{group} \glspl{resource} correspond to multiple \glspl{resource} whose write and read
payloads are packed into single packets. These \glspl{resource} typically have timing constraints
and are therefore put into a \gls{group}.

The \gls{group} send packet format consists of packed \gls{resource} write payloads as specified by
the Send Start and Send End indices from the \gls{group-member-discovery} packet in
Figure~\ref{fig:group-member-discovery-send-packet}.

\begin{figure}[h]
    \centering
    \begin{bytefield}[rightcurly=.]{16}
        \bitheader{0,7,8,15} \\
        \begin{leftwordgroup}{Resource Write}
            \begin{rightwordgroup}{Send Start$_i$}
                \bitbox{8}{Byte 0}
                \bitbox{8}{Byte 1}
            \end{rightwordgroup} \\
            \bitbox[]{6}{}
            \bitbox[]{4}{$\vdots$}
            \bitbox[]{6}{} \\
            [1ex]
            \begin{rightwordgroup}{Send End$_i$}
                \bitbox{8}{Byte $S(i) - 1$}
                \bitbox{8}{Byte $S(i)$}
            \end{rightwordgroup}
        \end{leftwordgroup} \\
        \wordbox[]{1}{$\vdots$} \\
        [1ex]
        \begin{leftwordgroup}{Resource Write}
            \begin{rightwordgroup}{Send Start$_N$}
                \bitbox{8}{\scriptsize Byte $S_S(i-1) + 1$}
                \bitbox{8}{\scriptsize Byte $S_S(i-1) + 2$}
            \end{rightwordgroup} \\
            \bitbox[]{6}{}
            \bitbox[]{4}{$\vdots$}
            \bitbox[]{6}{} \\
            [1ex]
            \begin{rightwordgroup}{Send End$_N$}
                \bitbox{8}{Byte $S_S(i) - 1$}
                \bitbox{8}{Byte $S_S(i)$}
            \end{rightwordgroup}
        \end{leftwordgroup}
    \end{bytefield}
    \begin{equation}
        S_S(n)=\sum_{i=1}^{n}{S(i)}
    \end{equation}
    \begin{equation}
        S(i) = \textrm{Send End}_i - \textrm{Send Start}_i
    \end{equation}
    \begin{equation}
        i \in [1, N], N = \textrm{Count}
    \end{equation}
    \caption{Group RPC send packet format.}
    \label{fig:group-rpc-send-packet-format}
\end{figure}

\FloatBarrier{}

The \gls{group} receive packet format consists of packed \gls{resource} read payloads as specified
by the Receive Start and Receive End indices from the \gls{group-member-discovery} packet in
Figure~\ref{fig:group-member-discovery-send-packet}.

\begin{figure}[h]
    \centering
    \begin{bytefield}[rightcurly=.]{16}
        \bitheader{0,7,8,15} \\
        \begin{leftwordgroup}{Resource Read}
            \begin{rightwordgroup}{Receive Start$_i$}
                \bitbox{8}{Byte 0}
                \bitbox{8}{Byte 1}
            \end{rightwordgroup} \\
            \bitbox[]{6}{}
            \bitbox[]{4}{$\vdots$}
            \bitbox[]{6}{} \\
            [1ex]
            \begin{rightwordgroup}{Receive End$_i$}
                \bitbox{8}{Byte $R(i) - 1$}
                \bitbox{8}{Byte $R(i)$}
            \end{rightwordgroup}
        \end{leftwordgroup} \\
        \wordbox[]{1}{$\vdots$} \\
        [1ex]
        \begin{leftwordgroup}{Resource Read}
            \begin{rightwordgroup}{Receive Start$_N$}
                \bitbox{8}{\scriptsize Byte $S_R(i-1) + 1$}
                \bitbox{8}{\scriptsize Byte $S_R(i-1) + 2$}
            \end{rightwordgroup} \\
            \bitbox[]{6}{}
            \bitbox[]{4}{$\vdots$}
            \bitbox[]{6}{} \\
            [1ex]
            \begin{rightwordgroup}{Receive End$_N$}
                \bitbox{8}{Byte $S_R(i) - 1$}
                \bitbox{8}{Byte $S_R(i)$}
            \end{rightwordgroup}
        \end{leftwordgroup}
    \end{bytefield}
    \begin{equation}
        S_R(n)=\sum_{i=1}^{n}{R(i)}
    \end{equation}
    \begin{equation}
        R(i) = \textrm{Receive End}_i - \textrm{Receive Start}_i
    \end{equation}
    \begin{equation}
        i \in [1, N], N = \textrm{Count}
    \end{equation}
    \caption{Group RPC receive packet format.}
    \label{fig:group-rpc-receive-packet-format}
\end{figure}

\FloatBarrier{}
\clearpage

\appendix
\section{Tables}

\FloatBarrier{}

\begin{table}[h]
    \centering
    \begin{tabular}{r|l}
        Status Code & Meaning \\
        \hline
        1 & Accepted \\
        2 & Rejected: Generic \\
        3 & Rejected: Unknown Resource \\
        4 & Rejected: Unknown Attachment \\
        5 & Rejected: Invalid Attachment \\
        6 & Rejected: Invalid Attachment Data \\
        7 & Rejected: Invalid Group ID \\
        8 & Rejected: Group Full \\
        9 & Rejected: Unknown Operation \\
        10 & Discard in progress \\
        11 & Discard complete \\
        12 & Rejected: Invalid Packet ID
    \end{tabular}
    \caption{Status codes.}
    \label{tab:status-codes}
\end{table}

\FloatBarrier{}
\clearpage

\printnoidxglossaries{}

\end{document}
